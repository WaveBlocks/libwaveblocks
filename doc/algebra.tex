\documentclass{article}

\usepackage[utf8]{inputenc} \usepackage[T1]{fontenc}

\usepackage{amsmath}
\usepackage{amsfonts}
\usepackage{amssymb}

%% Choose one of the following (if not choosing the default, 
%% viz., Computer Modern, font family):
 %\usepackage{lmodern}
 %\usepackage{mathpazo}
 %\usepackage{kpfonts}
 %\usepackage{mathptmx}
 %\usepackage{times,mtpro2}
 %\usepackage{stix}
 %\usepackage{txfonts}
 %\usepackage{newtxtext,newtxmath}
 %\usepackage{libertine} \usepackage[libertine]{newtxmath}

\usepackage{amsmath}
\usepackage{amssymb}
\usepackage{amsfonts}
\usepackage{amsopn}
\usepackage{braket}
\usepackage{bbm}
\usepackage{dsfont}
\usepackage{kpfonts}
\usepackage{amsthm}
% \usepackage{mathabx}

\parindent=0cm

% Various new commands that ease typesetting math even further
\newcommand{\dotp}[2]{\ensuremath{\langle #1 , #2 \rangle}}
\newcommand{\mat}[1]{\ensuremath{\mathbf{#1}}}
\newcommand{\mindex}[1]{\ensuremath{\underline{#1}}}

% EOF
  
\usepackage[ruled]{algorithm2e}

% Configure Algorithm2e
\DontPrintSemicolon
\SetKwInOut{Input}{Input}
\SetKwInOut{Output}{Output}

\begin{document}

\section{Ranks}

\begin{description}
\item[Basis Functions]
  \[ \phi(\vec{x}) \]
\item[Scalar Hagedorn Wavepacket]
  \[ \Phi^\varepsilon_{(\mathfrak{K},c)}[\Pi](\vec{x}) = \sum_{\mindex{k} \in \mathfrak{K}}
    c_{\mindex{k}}\phi_{\mindex{k}}(\vec{x}) \]
\item[Homogeneous Wavepacket]
  \[
    \Psi[\Pi](\vec{x}) =
    \begin{pmatrix}
      \Phi_1[\Pi](\vec{x}) \\
      \Phi_2[\Pi](\vec{x}) \\
    \end{pmatrix}
  \]
\item[Inhomogeneous Wavepacket]
  \[
    \Psi(\vec{x}) =
    \begin{pmatrix}
      \Phi_1[\Pi_1](\vec{x}) \\
      \Phi_2[\Pi_2](\vec{x}) \\
    \end{pmatrix}
  \]
\item[Linear Combination]
  \[
    \Upsilon(\vec{x}) = \alpha \Psi(\vec{x}) + \beta \Psi(\vec{y}) = 
    \begin{pmatrix}
      \alpha \Phi_{1,1}(\vec{x}) + \beta \Phi_{2,1}(\vec{x}) \\
      \alpha \Phi_{1,2}(\vec{x}) + \beta \Phi_{2,2}(\vec{x}) \\
    \end{pmatrix}
  \]
\end{description}


\section{Evaluation}

\[
  \Psi(\vec{x}) =
  \begin{pmatrix}
    \Phi_1(\vec{x}) \\
    \Phi_2(\vec{x}) \\
  \end{pmatrix}
\]

\section{Gradient}

\[
  \nabla \Phi_{\mathfrak{K}, c}^{\varepsilon}[\Pi](\vec{x}) =
  \begin{pmatrix}
    \Phi_{(\mathfrak{K}_{ext},\partial_1 c)}^{\varepsilon}[\Pi](\vec{x}) \\
    \Phi_{(\mathfrak{K}_{ext},\partial_2 c)}^{\varepsilon}[\Pi](\vec{x}) \\
  \end{pmatrix}
\]

\[
  \nabla \Psi =
  \begin{pmatrix}
    \nabla \Phi_1 \\
    \nabla \Phi_2 \\
  \end{pmatrix}
\]

\[
  \nabla \left(\alpha \Phi_1 + \beta \Phi_2 \right) =
  \alpha  \nabla \Phi_1 + \beta \nabla \Phi_2
\]


\end{document}