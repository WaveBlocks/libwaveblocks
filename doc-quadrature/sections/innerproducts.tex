\section{Inner Products}

In order to calculate wavepacket norms, various energies and other observables,
it is necessary to evaluate the braket
\begin{equation}
  \label{eq:braket}
  \mat{M} = \langle \Psi | F | \Psi' \rangle,
\end{equation}
where $\Psi$ and $\Psi'$ are wavepackets and $F$ specifies the operator.

For the mathematical derivation of the braket evaluation algorithms, again refer
to \cite{B_master_thesis}.


\subsection{Scalar Inner Products}

In the case of scalar wavepackets $\Phi$ and $\Phi'$ the braket is written as
\begin{equation}
  \mat{M} = \langle \Phi | f | \Phi' \rangle.
\end{equation}
The operator $f$ is given as a function parameter of the form
\begin{equation}
  f(\mat{x}, \vect{q}) : \mathbb{C}^{D \times R} \times \mathbb{R}^D \rightarrow
  \mathbb{C}^R
\end{equation}
mapping $D$-dimensional nodal points $\mat{x}$ and position
$\vect{q}$ to the operator's values.

Here, $R$ is the order of the quadrature.
Instead of calling the operator function for the different nodes separately, all
points are given to the function at once to make efficient vectorized
calculations in its definition possible.

The inner product calculation implementation provides two different methods,
\texttt{build\_matrix} and \texttt{quadrature}.
The first returns a $|\mathfrak{K}| \times |\mathfrak{K'}|$-sized matrix, one
entry for each pair of basis functions of $\Phi$ and $\Phi'$, where
$\mathfrak{K}$ is the basis shape size.
\texttt{quadrature} reduces the matrix to a scalar by factoring in the
coefficient vectors $\vect{c}$ and $\vect{c}'$ of $\Phi$ and $\Phi'$, returning
$\vect{c}^H \mat{M} \vect{c}'$.

The ``inhomogeneous'' evaluation code for different wavepackets $\Phi$ and
$\Phi'$ can be found in \path{waveblocks/inhomogeneous_inner_product.hpp}.
If the two wavepackets are the equivalent ($\Phi = \Phi'$), a more efficient
``homogeneous'' algorithm can be used instead which is implemented in
\path{waveblocks/homogeneous_inner_product.hpp}


\subsection{Multi-Component Inner Products}

In the more general case the wavepackets can have multiple components:
\begin{align}
  \Psi &= \begin{pmatrix} \Phi_1 \\ \vdots \\ \Phi_N \end{pmatrix} \\
  \mat{M} &= \langle \Psi | F | \Psi' \rangle = \left\langle
    \begin{pmatrix} \Phi_1 \\ \vdots \\ \Phi_N \end{pmatrix} \left|
    \begin{pmatrix} \ddots & \vdots & \udots \\ \cdots & f_{i,j} & \cdots \\
      \udots & \vdots & \ddots \end{pmatrix} \right|
    \begin{pmatrix} \Phi'_1 \\ \vdots \\ \Phi'_M \end{pmatrix}
    \right\rangle
\end{align}

The calculation of these brakets is built on top of the inhomogeneous scalar
evaluation code, calling it for every pair of wavepackets $(\Phi_i,\Phi'_j)$ and
corresponding function $f_{i,j}$.

The source code for this algorithm is found in
\path{waveblocks/vector_inner_product.hpp}.
