\section{Results}
\label{sec:results}

Since one of the main reasons of porting WaveBlocksND from Python to C++ was to
increase its performance, run-times for various use-cases must be measured and
compared to evaluate the improvements.
All tests were done using homogeneous quadrature with an identity operator;
they can be found in the file \path{test/test_innerproduct_runtimes.cpp}.
The computer used for benchmarking had a quad-core Intel i5-2500 processor.

Different parameters were varied to compare the scaling behavior of the two
implementations.
Because the algorithms used in both cases were essentially the same, the
run-times grow with the parameters roughly at the same rate, however there is a
big difference in the absolute durations.

\begin{figure}
  \center
  \input{figures/speedup-one-dim.tex}
  \caption{Run-times for 1-D homogeneous quadrature of order 8.}
  \label{fig:speedup1d}
\end{figure}

In the simplest test, the number of coefficients of a wavepacket was changed
for a one-dimensional, single-component quadrature.
The results in figure~\ref{fig:speedup1d} show a huge improvement in the C++
version with a speed-up of factor 50 for smaller problems, converging to a
factor of about 9 for larger ones.

\begin{figure}
  \center
  \input{figures/speedup-multi-dim.tex}
  \caption{Run-times for multi-dimensional homogeneous quadrature of order 8
    with 10 coefficients per dimension.}
  \label{fig:speedupnd}
\end{figure}

The next test inspects the impact of dimensionality using tensor product
quadrature.
As can be seen from figure~\ref{fig:speedupnd}, higher-dimensional problems can
have a 20-fold speed-up.

\begin{figure}
  \center
  \input{figures/speedup-genz-keister.tex}
  \caption{Run-times for multi-dimensional homogeneous Genz-Keister quadrature
    of level 6 with 10 coefficients per dimension.}
  \label{fig:speedupgenzkeister}
\end{figure}

Figure~\ref{fig:speedupgenzkeister} results from the same type of test as above,
but with Genz-Keister quadrature of level 6 instead of dense tensor products.
The speed-up is in a similar range.

\begin{figure}
  \center
  \input{figures/speedup-multi-comp.tex}
  \caption{Run-times for multi-component 2-D homogeneous quadrature of order 8
    with 10 coefficients per dimension.}
  \label{fig:speedupncomps}
\end{figure}

Finally, figure~\ref{fig:speedupncomps} shows the behavior for varying sizes
of multi-component wavepackets in two dimensions.
It is apparent that the speed-up is very consistently a factor of 75.

In conclusion, there is no doubt that replacing the quadrature code with a C++
version using the highly-optimized Eigen library brings great improvements.
However, it must be noted that these are isolated measurements and the speed-up
is only significant if a large part of the execution time is spent in the
quadrature.
This means that for smaller problems, even though the measured speed-ups are
huge, there are probably bottlenecks in other parts of the code, which
diminishes the total improvement.
Additionally, a noticeable trade-off is a much longer compilation time, which
can hinder doing many test with small problems.
